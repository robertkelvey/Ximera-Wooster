\documentclass{ximera}
\input{../preamble.tex}
\title{Differentiation Rules Part Two}
\begin{abstract}
\end{abstract}
\begin{document}
\maketitle
\begin{javascript}
 caseInsensitive = function(a,b) {
    return a.toLowerCase() == b.toLowerCase();
  };
\end{javascript}
\begin{dialogue}
\item[Julia] You know, some of those rules we learned were pretty useful, but some of these derivatives still suck! There \textbf{HAS} to be a better way!
\item[Dylan] I'm sure there is, and I'm sure I know who could help us!
\item[James] Did I hear my name?
\item[Dylan] Not yet!
\item[Julia] James!
\item[James] There are more rules for differentiation that can make your life just a little bit easier!
\end{dialogue}
\section{The Product Rule}
\begin{dialogue}
\item[James] From the last time we did this, what rule do you think would exist for the product of two functions?
\item[Julia] Well, last time we added or subtracted the derivative of both functions, so I bet we multiply the derivative of both!
\item[Dylan] Let's check!
\end{dialogue}
Consider the functions $f(x) = 2x$ and $g(x) = 3x^3 + x^2$.
\[
\graph{f(x)=2x, g(x)=3x^3+x^2}
\]
\begin{question}
Use Julia's guess to find the derivative of $f(x) \cdot g(x)$.

$\answer{18x^2+4x}$

\begin{definition}
  The \textbf{derivative} of $f(x)$ at $a$ is defined by the following limit:
  \[
  \eval{\frac{d}{dx} f(x)}_{x=a} = \lim_{h\to 0} \frac{f(a+h) - f(a)}{h}.
  \]
\end{definition}

Use the limit definition of the derivative to find the derivative of $f(x) \cdot g(x)$.

$\answer{24x^3+6x^2}$

Was Julia right?
\begin{multipleChoice}
\choice{Yes}
\choice[correct]{No}
\end{multipleChoice}
\end{question}
\begin{dialogue}
\item[Julia] Darn! It didn't work!
\item[Dylan] It must be a little harder than that...
\item[James] That's right Dylan, but it is easier than the limit definition! All we have to do is use $$\frac{d}{dx}\left(f(x)\cdot g(x)\right)= f(x)\cdot g'(x) + f'(x)\cdot g(x)\text{.}$$ This is called the \textbf{Product Rule}.
\end{dialogue}
\begin{question}
Using the Product Rule, differentiate the products of the following functions:

$f(x) = 6x^3$, $g(x) = 7x^4$

$\answer{294x^6}$

$f(x) = \cos(x)+4x$, $g(x) = 3x^2$

$\answer{-3x^2\sin(x)+6x\cos(x)}$

$f(x) = x^2$, $g(x) = 3x^3-3x$

$\answer{15x^4-9x^2}$

$f(x) = x^7$, $g(x) = 2x^{32}$

$\answer{78x^{38}}$

\end{question}

\section{The Quotient Rule}
\begin{dialogue}
\item[Dylan] Wow! That's gonna save a ton of time with products! Is there anything like it we can do with quotients?
\item[James] There is! It's even called \textbf{the Quotient Rule}!
\item[Julia] I bet it's a pain too though, just like the product rule.
\item[James] Well, why don't you try using your intuition first rather than guessing?
\item[Dylan] Alright, well, I guess I would divide the derivative of the numerator by the derivative of the denominator.
\end{dialogue}
\begin{question}
Consider the functions $f(x) = x^3+1$ and $g(x) = x$.
\[
\graph{f(x)=x^3+1, g(x)=x}
\]
Use Dylan's guess to find the derivative of $\frac{f(x)}{g(x)}$.

$\answer{3x^2/1}$

Use the limit definition of the derivative to find the derivative of $\frac{f(x)}{g(x)}$.

$\answer{(2x^3-1)/x^2}$

Was Dylan right?

\begin{multipleChoice}
\choice{Yes}
\choice[correct]{No}
\end{multipleChoice}
\end{question}
\begin{dialogue}
\item[Julia] I knew it! It's never that easy!
\item[James] Now calm down Julia, this rule is worse than the last one, but it's much better than going through by the limit definition:  $$\frac{d}{dx}\left( \frac{f(x)}{g(x)}\right) = \frac{f'(x)g(x)-f(x)g'(x)}{g(x)^2}\text{.}$$
\end{dialogue}
\begin{question}
Using the Quotient Rule, differentiate the products of the following functions to find $\frac{d}{dx}\eval{\dfrac{f(x)}{g(x)}}$:

$f(x) = 3x-1$, $g(x) = 2x+1$

$\answer{5/(2x+1)^2}$

$f(x) = 1$, $g(x) = x+10$

$\answer{-1/(x+10)^2}$

$f(x) = x^2$, $g(x) = 3x^3-3x$

$\answer{(-x^2-1)/(3(x^2-1)^2)}$

$f(x) = x^7$, $g(x) = 2x^{32}$

$\answer{-25/(2x^{26})}$
\end{question}
\section{The Chain Rule}
\begin{dialogue}
\item[James] There's one last rule to learn today; the \textbf{Chain Rule}.
\item[Dylan] That rule sounds pretty cool! When do we use it though? I thought we already covered the functions we need to know...
\item[Julia] Yeah, what else is there?
\item[James] We use the chain rule in composition of functions, like when we have $\sin(2x)$ - $2x$ is a function, and so is $\sin(x)$
\item[Julia] And how bad is the rule?
\item[James] This one is a little more tricky - $$\frac{d}{dx}f(g(x)) = f'(g(x))\cdot g'(x)\text{.}$$
\item[Dylan and Julia] That's so gross.
\item[James] Well, let's give it a try and see if you like it more than the limit definition!
\end{dialogue}
\begin{question}
Consider $f(x) = \sqrt(x)$ and $g(x) = \frac{1}{x}$
\[
\graph{sqrt(x), 1/x}
\]

Using the limit definition of derivative, evaluate the derivative of $f(g(x))$.

$\answer{-1/2x^{-3/2}}$

Now, evaluate the same limit using the chain rule. Notice you get the same answer. Yay.

\end{question}
\begin{question}

Find the compostition $f(g(x))$, then using the Chain Rule, differentiate $f(g(x))$ for the following functions:

$f(x) = 3x+x^2$, $g(x) = x^4+7x$

$\begin{prompt}f(g(x))=\answer{3(x^4+7x)+(x^4+7x)^2}\end{prompt}$
$\begin{prompt}\dfrac{d}{dx}\eval{f(g(x))}=\answer{8x^7+70x^4+12x^3+98x+21}\end{prompt}$

$f(x) = \cos(x)$, $g(x) = \sin(x)$

$\begin{prompt}f(g(x))=\answer{\cos(\sin(x))}\end{prompt}$
$\begin{prompt}\dfrac{d}{dx}\eval{f(g(x))}=\answer{-\cos(x)\sin(\sin(x))}\end{prompt}$

$f(x) = \cos(x)$, $g(x) = x^3$

$\begin{prompt}f(g(x))=\answer{\cos(x^3)}\end{prompt}$
$\begin{prompt}\dfrac{d}{dx}\eval{f(g(x))}=\answer{-3x^2\sin(x^3)}\end{prompt}$

\end{question}

\begin{question}

Using the Chain Rule, differentiate the compositions $g(f(x))$ for the following functions:

$f(x) = 3x+x^2$, $g(x) = x^4+7x$

$\begin{prompt}g(f(x))=\answer{(3x+x^2)^4+7(3x+x^2)}\end{prompt}$
$\begin{prompt}\dfrac{d}{dx}\eval{g(f(x))}=\answer{(2x+3)(4x^3(x+3)^3+7)}\end{prompt}$

$f(x) = \cos(x)$, $g(x) = \sin(x)$

$\begin{prompt}g(f(x))=\answer{\sin(\cos(x))}\end{prompt}$
$\begin{prompt}\dfrac{d}{dx}\eval{g(f(x))}=\answer{-\cos(\cos(x))\sin(x)}\end{prompt}$

$f(x) = \cos(x)$, $g(x) = x^3$

$\begin{prompt}g(f(x))=\answer{\cos(x)^3}\end{prompt}$
$\begin{prompt}\dfrac{d}{dx}\eval{g(f(x))}=\answer{-3\cos^2(x)\sin(x)}\end{prompt}$

\end{question}

\section{In Summary}
We've covered a lot of differentiation rules in this lab, to help you out we've summarized the theorems below:
\begin{theorem}[The Product Rule]
If $f$ and $g$ are differentiable, then
\[
\dfrac{d}{dx}\left[f(x)g(x)\right]=f(x)\dfrac{d}{dx}\left[g(x)\right]+g(x)\dfrac{d}{dx}\left[f(x)\right]
\]
\end{theorem}
\begin{theorem}[The Quotient Rule]
If $f$ and $g$ are differentiable, then
\[
\dfrac{d}{dx}\left[\dfrac{f(x)}{g(x)}\right]=\dfrac{g(x)\ddx\left[f(x)\right]-f(x)\ddx\left[g(x)\right]}{\left[g(x)\right]^2}
\]
\end{theorem}
\begin{theorem}[The Chain Rule]
\[
\dfrac{d}{dx}\left[f(x)g(x)\right]=f(x)\dfrac{d}{dx}\left[g(x)\right]+g(x)\dfrac{d}{dx}\left[f(x)\right]
\]
\end{theorem}

\end{document}
